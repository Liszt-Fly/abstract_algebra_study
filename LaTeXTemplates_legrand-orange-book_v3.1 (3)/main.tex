
\documentclass[
	11pt, % Default font size, select one of 10pt, 11pt or 12pt
	fleqn, % Left align equations
	a4paper, % Paper size, use either 'a4paper' for A4 size or 'letterpaper' for US letter size
	%oneside, % Uncomment for oneside mode, this doesn't start new chapters and parts on odd pages (adding an empty page if required), this mode is more suitable if the book is to be read on a screen instead of printed
]{LegrandOrangeBook}
\usepackage{xeCJK}
\usepackage{fontspec}
\usepackage{ctex}
\usepackage{color}
\setmainfont[]{PingFang HK}
\setCJKmainfont{Heiti SC}
% Book information for PDF metadata, remove/comment this block if not required
\hypersetup{
	pdftitle={Title}, % Title field
	pdfauthor={Author}, % Author field
	pdfsubject={Subject}, % Subject field
	pdfkeywords={Keyword1, Keyword2, ...}, % Keywords
	pdfcreator={LaTeX}, % Content creator field
}

\addbibresource{sample.bib} % Bibliography file

\definecolor{ocre}{RGB}{243, 102, 25} % Define the color used for highlighting throughout the book

\chapterimage{orange1.jpg} % Chapter heading image
\chapterspaceabove{6.5cm} % Default whitespace from the top of the page to the chapter title on chapter pages
\chapterspacebelow{6.75cm} % Default amount of vertical whitespace from the top margin to the start of the text on chapter pages


\begin{document}

\titlepage % Output the title page
{\includegraphics[width=\paperwidth]{background.pdf}} % Code to output the background image, which should be the same dimensions as the paper to fill the page entirely; leave empty for no background image
{ % Title(s) and author(s)
	\centering\sffamily % Font styling
	{\Huge\bfseries Exploring the Physical Manifestation of Humanity's Subconscious Desires\par} % Book title
	\vspace{16pt} % Vertical whitespace
	{\LARGE A Practical Guide\par} % Subtitle
	\vspace{24pt} % Vertical whitespace
	{\huge\bfseries Goro Akechi\par} % Author name
}

%----------------------------------------------------------------------------------------
%	COPYRIGHT PAGE
%----------------------------------------------------------------------------------------

\thispagestyle{empty} % Suppress headers and footers on this page

~\vfill % Push the text down to the bottom of the page

\noindent Copyright \copyright\ 2022 Goro Akechi\\ % Copyright notice

\noindent \textsc{Published by Publisher}\\ % Publisher

\noindent \textsc{\href{https://www.latextemplates.com/template/legrand-orange-book}{book-website.com}}\\ % URL

\noindent Licensed under the Creative Commons Attribution-NonCommercial 4.0 License (the ``License''). You may not use this file except in compliance with the License. You may obtain a copy of the License at \url{https://creativecommons.org/licenses/by-nc-sa/4.0}. Unless required by applicable law or agreed to in writing, software distributed under the License is distributed on an \textsc{``as is'' basis, without warranties or conditions of any kind}, either express or implied. See the License for the specific language governing permissions and limitations under the License.\\ % License information, replace this with your own license (if any)

\noindent \textit{First printing, March 2022} % Printing/edition date

%----------------------------------------------------------------------------------------
%	TABLE OF CONTENTS
%----------------------------------------------------------------------------------------

\pagestyle{empty} % Disable headers and footers for the following pages

\tableofcontents % Output the table of contents

\listoffigures % Output the list of figures, comment or remove this command if not required

\listoftables % Output the list of tables, comment or remove this command if not required

\pagestyle{fancy} % Enable default headers and footers again

\cleardoublepage % Start the following content on a new page

%----------------------------------------------------------------------------------------
%	PART
%----------------------------------------------------------------------------------------

\part{Part One Title}

%----------------------------------------------------------------------------------------
%	SECTIONING EXAMPLES CHAPTER
%----------------------------------------------------------------------------------------

\chapterimage{orange2.jpg} % Chapter heading image
\chapterspaceabove{6.75cm} % Whitespace from the top of the page to the chapter title on chapter pages
\chapterspacebelow{7.25cm} % Amount of vertical whitespace from the top margin to the start of the text on chapter pages

%------------------------------------------------

\chapter{Pre-requisite knowledge:Set Theory}

\section{Explanation of symbols}
在抽象代数的学习中,会使用到集合中的知识,当然也会使用到集合中的各种符号,在这一小节中,对这些符号进行一遍复习。
\subsection{不同数的符号表示}
\begin{itemize}
	\item $\mathbb{R}$ 全体实数(有理数和无理数)的集合
	\item $\mathbb{N}$ 全体自然数集合
	\item $\mathbb{N^*}$ 全体非负整数排除0的集合
	\item $\mathbb{Q}$ 全体有理数(整数和分数)的集合
	\item $\mathbb{Z}$ 全体整数的集合
	\item $\mathbb{C}$ 全体复数的集合
\end{itemize}

\subsection{集合中的常见概念符号}
\begin{itemize}
	\item 集合 A,B,C
	\item 元素 a,b,c
	\item 空集 $\emptyset$
	\item 元素与集合之间的从属关系:$\in$,$\notin$
	\item 集合与集合之间的从属关系:$\subset$,$\subseteq$,$\not\subset$
	\item 交集,补集,并集:$A\cup B$ $A\cap B$,$A^\complement$
\end{itemize}

\subsection{集合中的常见概念}
要证明两个集合相等,只需要证明这两个集合相互包含即可,这是一个非常常见的证明集合相等的手段。
\begin{theorem}
	定理一:两个集合A,B相等的充要条件:$A=B\Longleftrightarrow A\subset B\Longleftrightarrow B\subset A$
\end{theorem}
在抽象代数中,我们将使用集合的笛卡尔积来定义映射,这是一个非常重要的概念。
\begin{theorem}
	我们称:$A_1\times A_2\times ...\times A_n=\{(a_1,a_2,...,a_n)|a_i\in A_i\}$为n个集合$A_1,A_2,...A_n$的笛卡尔积。
\end{theorem}
()内表示的是有序数组,而$a_1$则表示的是$A_1$里的元素。
\begin{example}
	$$
		A=\{a,b,c\},B=\{1,2\},则
		A\times B =\{(a,1),(b,1),(c,1),(a,2),(b,2),(c,2)\}
	$$
\end{example}
根据上面这个例题,我们可以推断出一个结论。
\begin{theorem}
	一般地,如果$|A|=m,|B|=n$那么$|A\times B|=mn$
\end{theorem}
\begin{remark}
	一般,我们使用$|A|$来表示A集合中元素的个数。比如上面的$A\times B$表示的就是AB笛卡尔积所构成的集合的元素的个数。
\end{remark}

\section{mapping}
\begin{theorem}{\emph{映射的定义}}

	设$\emptyset$是从笛卡尔积$A_1\times A_2 \times ...\times A_n$到集合D的一个法则,如果$A_1\times A_2 \times ...\times A_n$中的每一个元素($a_1,a_2,...a_n$)都有D中唯一的元素d与之对应,那么我们称$\emptyset$是从$A_1\times A_2 \times ...\times A_n$到D的一个映射。
\end{theorem}
\begin{example}
	设$A_1=\{\mbox{东,西}\},A_2=\{\mbox{南}\},D=\{\mbox{高,低}\}$,
	则$\emptyset_1$(西,南)=高不是$A_1\times A_2$到D的映射,因为只定义了一种情况,总共有两种情况,没有进行一一对应。如果改为$\emptyset_2$(西,南)=高,$\emptyset_2$(东,南)=低,符合定义,所以是$A_1\times A_2$到D的映射。
\end{example}

\begin{example}
	设$A_1=D=\mathbb{R},则$

	$\emptyset(a)=a,a\not =1$\par
	$\emptyset(1)=b,b^2=1$\par
	不是$A_1$到D的映射。\par
	虽然这个映射对每一个定义域内的变量都进行了映射,但是在自变量为1的时候b可以等于+1也可以等于-1,不符合一一对应的条件,所以不是映射。
\end{example}

\begin{example}
	设$A_1=D=\mathbb{Z}_+$,则\par
	$\emptyset(a)=a-1$\par
	不是$A_1$到D的映射。\par
	由于A和D都是属于正整数集合,所以当a=1的时候映射结果不在D集合内,所以不是。
\end{example}

\begin{theorem}{映射相等}

	设$\emptyset_1,\emptyset_2$都是从笛卡尔积$A_1\times A_2 \times ...\times A_n$到集合D的映射,如果对于$A_1\times A_2 \times ...\times A_n$中的每一个元素($a_1,a_2,...,a_n$)都有
	\begin{center}
		$\emptyset_1(a_1,a_2,...,a_n)=\emptyset_2(a_1,a_2,...,a_n)$,
	\end{center}\par
	则称这两个映射$\emptyset_1,\emptyset_2$是相等的。
\end{theorem}
\begin{remark}
	特别注意:两个映射相等,实际上的要求是:
	\begin{itemize}
		\item 它们的定义域相等
		\item 它们的作用效果是相通的
	\end{itemize}
\end{remark}

\begin{example}
	设A=D都表示正整数的集合,$\emptyset_1: A\rightarrow D$ 定义为:$\emptyset_1(a)=1,\emptyset_2: A\rightarrow $定义为:$\emptyset_2(a)=a^0$,则$\emptyset_1=\emptyset_2$

	由于前提条件是正整数的集合,而不是自然数,元素自然不可能是0,所以定义域一样,而作用效果也一样,映射的结果都是1,所以映射是相等的,但是如果将本例改为自然数,则是错误的。
\end{example}
\section{algebraic operation}
\subsection{Basic Algebraic Operation}
本节的目标任务就是重新定义代数运算,打破之前对代数运算的认知,从映射与集合的观点来重新定义。

\begin{theorem}
	一个从$A\times B$到D的映射叫做一个$A\times B$到D的代数运算。
\end{theorem}


映射运算$\emptyset$:$A\times B\rightarrow D,(a,b)\rightarrow d=\emptyset(a,b)$ \par
代数运算$\circ$: $A\times B \rightarrow D,(a,b)\rightarrow d=a \circ b$

\begin{example}
	设$A=\mathbb{Z},B=\mathbb{Z}-\{0\},D=\mathbb{Q}$,则
	\begin{center}
		$\circ:(a,b)\rightarrow \frac{a}{b}=a\circ b$
	\end{center}
	是一个$A\times B$到D的代数运算,也就是普通的除法。

	\textcolor{Salmon}{对于被除数是整数,除数是不为0的整数来说,除法的运算结果既有可能是分数,也有可能是不为0的整数,所以综合来看映射自然是有理数集。}

\end{example}

\begin{example}
	设A=\{1,2\},B=\{1,2\},D=\{奇,偶\},则
	\begin{center}
		$\circ:(1,1)\rightarrow$奇,
		$(1,2)\rightarrow$奇,
		$(2,1)\rightarrow$偶,
		$(2,2)\rightarrow$偶
	\end{center}
	是一个$A\times B$到D的代数运算。

	\textcolor{Salmon}{对于$A\times$B 笛卡尔积的每一个有序数组都规定了映射的结果,并且结果都在D集合中存在,所以自然符合代数运算的含义。}
\end{example}

\begin{theorem}
	我们称$A\times A$的代数运算$\circ$为A上的代数运算,或者A上的二元运算。$\circ$具有封闭性。
\end{theorem}

\begin{example}
	设$A=\mathbb{Z}$,则普通数的加法,减法,乘法,都是集合A上的代数运算。

	\textcolor{Salmon}{结论当然是成立的,任意一个整数+整数,整数-整数,整数*整数,结果都是整数。}
\end{example}

\subsection{operational rule}

\begin{theorem}
	设$circ$是集合A上的一个代数运算。
	\begin{itemize}
		\item 如果对于$\forall a,b,c\in A$,都有$(a\circ b)\circ c=a\circ(b\circ c)$,则称$\circ$适合结合律。
		\item 如果对于$\forall a,b\in A$,都有$a\circ b=b\circ a$,则称$\circ$适合交换律。
	\end{itemize}
\end{theorem}

\begin{example}
	在有理数集$\mathbb{Q}$上规定代数运算$\circ$为普通加法+,那么显然$\circ$适合结合律和交换律,并且显然有:
	\begin{center}
		$[(1\circ 2)\circ(-1)]\circ(-2)=0$,

		$1\circ\{[2\circ(-1)]\circ(-2)\}=0$,

		$\{[(-2)\circ 2]\circ 1\}\circ(-1)=0$.
	\end{center}
从这个例子可以看出,当$\circ$适合结合律和交换律的时候,\textcolor{Salmon}{任意方式加括号不改变若干个元素的乘积,任意改变元素的顺序也不改变若干元素的乘积。}
\end{example}

\chapter{In-text Element Examples}

\section{Referencing Publications}\index{Citation}

This statement requires citation \cite{Smith:2022jd}; this one is more specific \cite[162]{Smith:2021qr}.

%------------------------------------------------

\section{Link Examples}\index{Links}

This is a URL link: \href{https://www.latextemplates.com}{LaTeX Templates}. This is an email link: \href{mailto:example@example.com}{example@example.com}. This is a monospaced URL link: \url{https://www.LaTeXTemplates.com}.

%------------------------------------------------

\section{Lists}\index{Lists}

Lists are useful to present information in a concise and/or ordered way.

\subsection{Numbered List}\index{Lists!Numbered List}

\begin{enumerate}
	\item First numbered item
	      \begin{enumerate}
		      \item First indented numbered item
		      \item Second indented numbered item
		            \begin{enumerate}
			            \item First second-level indented numbered item
		            \end{enumerate}
	      \end{enumerate}
	\item Second numbered item
	\item Third numbered item
\end{enumerate}

\subsection{Bullet Point List}\index{Lists!Bullet Points}

\begin{itemize}
	\item First bullet point item
	      \begin{itemize}
		      \item First indented bullet point item
		      \item Second indented bullet point item
		            \begin{itemize}
			            \item First second-level indented bullet point item
		            \end{itemize}
	      \end{itemize}
	\item Second bullet point item
	\item Third bullet point item
\end{itemize}

\subsection{Descriptions and Definitions}\index{Lists!Descriptions and Definitions}

\begin{description}
	\item[Name] Description
	\item[Word] Definition
	\item[Comment] Elaboration
\end{description}

%------------------------------------------------

\section{International Support}

àáâäãåèéêëìíîïòóôöõøùúûüÿýñçčšž

\noindent ÀÁÂÄÃÅÈÉÊËÌÍÎÏÒÓÔÖÕØÙÚÛÜŸÝÑ

\noindent ßÇŒÆČŠŽ

%------------------------------------------------

\section{Ligatures}

fi fj fl ffl ffi Ty

%----------------------------------------------------------------------------------------
%	PART
%----------------------------------------------------------------------------------------

\part{Part Two Title}

%----------------------------------------------------------------------------------------
%	MATHEMATICS EXAMPLES CHAPTER
%----------------------------------------------------------------------------------------

\chapter{Mathematics}

\section{Theorems}\index{Theorems}

\subsection{Several equations}\index{Theorems!Several Equations}

This is a theorem consisting of several equations.

\begin{theorem}[Name of the theorem] % Specify a name/title in square brackets, or leave them out for no title
	In $E=\mathbb{R}^n$ all norms are equivalent. It has the properties:
	\begin{align}
		 & \big| ||\mathbf{x}|| - ||\mathbf{y}|| \big|\leq || \mathbf{x}- \mathbf{y}||                            \\
		 & ||\sum_{i=1}^n\mathbf{x}_i||\leq \sum_{i=1}^n||\mathbf{x}_i||\quad\text{where $n$ is a finite integer}
	\end{align}
\end{theorem}

\subsection{Single Line}\index{Theorems!Single Line}

This is a theorem consisting of just one line.

\begin{theorem} % Specify a name/title in square brackets, or leave them out for no title
	A set $\mathcal{D}(G)$ in dense in $L^2(G)$, $|\cdot|_0$.
\end{theorem}

%------------------------------------------------

\section{Definitions}\index{Definitions}

A definition can be mathematical or it could define a concept.

\begin{definition}[Definition name] % Specify a name/title in square brackets, or leave them out for no title
	Given a vector space $E$, a norm on $E$ is an application, denoted $||\cdot||$, $E$ in $\mathbb{R}^+=[0,+\infty[$ such that:
	\begin{align}
		 & ||\mathbf{x}||=0\ \Rightarrow\ \mathbf{x}=\mathbf{0}        \\
		 & ||\lambda \mathbf{x}||=|\lambda|\cdot ||\mathbf{x}||        \\
		 & ||\mathbf{x}+\mathbf{y}||\leq ||\mathbf{x}||+||\mathbf{y}||
	\end{align}
\end{definition}

%------------------------------------------------

\section{Notations}\index{Notations}

\begin{notation} % Specify a name/title in square brackets, or leave them out for no title
	Given an open subset $G$ of $\mathbb{R}^n$, the set of functions $\varphi$ are:
	\begin{enumerate}
		\item Bounded support $G$;
		\item Infinitely differentiable;
	\end{enumerate}
	a vector space is denoted by $\mathcal{D}(G)$.
\end{notation}

%------------------------------------------------

\section{Remarks}\index{Remarks}

This is an example of a remark.

\begin{remark}
	The concepts presented here are now in conventional employment in mathematics. Vector spaces are taken over the field $\mathbb{K}=\mathbb{R}$, however, established properties are easily extended to $\mathbb{K}=\mathbb{C}$.
\end{remark}

%------------------------------------------------

\section{Corollaries}\index{Corollaries}

\begin{corollary}[Corollary name] % Specify a name/title in square brackets, or leave them out for no title
	The concepts presented here are now in conventional employment in mathematics. Vector spaces are taken over the field $\mathbb{K}=\mathbb{R}$, however, established properties are easily extended to $\mathbb{K}=\mathbb{C}$.
\end{corollary}

%------------------------------------------------

\section{Propositions}\index{Propositions}

\subsection{Several equations}\index{Propositions!Several Equations}

\begin{proposition}[Proposition name] % Specify a name/title in square brackets, or leave them out for no title
	It has the properties:
	\begin{align}
		 & \big| ||\mathbf{x}|| - ||\mathbf{y}|| \big|\leq || \mathbf{x}- \mathbf{y}||                            \\
		 & ||\sum_{i=1}^n\mathbf{x}_i||\leq \sum_{i=1}^n||\mathbf{x}_i||\quad\text{where $n$ is a finite integer}
	\end{align}
\end{proposition}

\subsection{Single Line}\index{Propositions!Single Line}

\begin{proposition} % Specify a name/title in square brackets, or leave them out for no title
	Let $f,g\in L^2(G)$; if $\forall \varphi\in\mathcal{D}(G)$, $(f,\varphi)_0=(g,\varphi)_0$ then $f = g$.
\end{proposition}

%------------------------------------------------

\section{Examples}\index{Examples}

\subsection{Equation Example}\index{Examples!Equation}

\begin{example} % Specify a name/title in square brackets, or leave them out for no title
	Let $G=\{x\in\mathbb{R}^2:|x|<3\}$ and denoted by: $x^0=(1,1)$; consider the function:
	\begin{equation}
		f(x)=\left\{\begin{aligned}                                                                                                                                    & \mathrm{e}^{|x|} &  & \text{si $|x-x^0|\leq 1/2$} \\
                                                                                                                                                   & 0                &  & \text{si $|x-x^0|> 1/2$}\end{aligned}\right.
	\end{equation}
	The function $f$ has bounded support, we can take $A=\{x\in\mathbb{R}^2:|x-x^0|\leq 1/2+\epsilon\}$ for all $\epsilon\in\mathopen{]}0\,;5/2-\sqrt{2}\mathclose{[}$.
\end{example}

\subsection{Text Example}\index{Examples!Text}

\begin{example}[Example name] % Specify a name/title in square brackets, or leave them out for no title
	Aliquam arcu turpis, ultrices sed luctus ac, vehicula id metus. Morbi eu feugiat velit, et tempus augue. Proin ac mattis tortor. Donec tincidunt, ante rhoncus luctus semper, arcu lorem lobortis justo, nec convallis ante quam quis lectus. Aenean tincidunt sodales massa, et hendrerit tellus mattis ac. Sed non pretium nibh. Donec cursus maximus luctus. Vivamus lobortis eros et massa porta porttitor.
\end{example}

%------------------------------------------------

\section{Exercises}\index{Exercises}

\begin{exercise} % Specify a name/title in square brackets, or leave them out for no title
	This is a good place to ask a question to test learning progress or further cement ideas into students' minds.
\end{exercise}

%------------------------------------------------

\section{Problems}\index{Problems}

\begin{problem} % Specify a name/title in square brackets, or leave them out for no title
What is the average airspeed velocity of an unladen swallow?
\end{problem}

%------------------------------------------------

\section{Vocabulary}\index{Vocabulary}

Define a word to improve a students' vocabulary.

\begin{vocabulary}[Word] % Specify a name/title in square brackets, or leave them out for no title
	Definition of word.
\end{vocabulary}

%----------------------------------------------------------------------------------------
%	PRESENTING INFORMATION/RESULTS EXAMPLES CHAPTER
%----------------------------------------------------------------------------------------

\chapterimage{orange3.jpg} % Chapter heading image
\chapterspaceabove{6.25cm} % Whitespace from the top of the page to the chapter title on chapter pages
\chapterspacebelow{7.5cm} % Amount of vertical whitespace from the top margin to the start of the text on chapter pages

%------------------------------------------------

\chapter{Presenting Information and Results with a Long Chapter Title}

\section{Table}\index{Table}

Lorem ipsum dolor sit amet, consectetur adipiscing elit. Praesent porttitor arcu luctus, imperdiet urna iaculis, mattis eros. Pellentesque iaculis odio vel nisl ullamcorper, nec faucibus ipsum molestie. Sed dictum nisl non aliquet porttitor. Etiam vulputate arcu dignissim, finibus sem et, viverra nisl. Aenean luctus congue massa, ut laoreet metus ornare in. Nunc fermentum nisi imperdiet lectus tincidunt vestibulum at ac elit. Nulla mattis nisl eu malesuada suscipit.

\begin{table}[H] % Use [H] to suppress floating and place the figure/table exactly where it is specified in the text
	\centering % Horizontally center the table on the page
	\begin{tabular}{L{0.15\textwidth} R{0.15\textwidth} R{0.15\textwidth}} % Specify column alignment with L{width}, C{width} and R{width} for fixed-width columns, or the default latex l, c and r for flexible-width columns
		\toprule
		\textbf{Treatments} & \textbf{Response 1} & \textbf{Response 2} \\
		\midrule
		Treatment 1         & 0.0003262           & 0.562               \\
		Treatment 2         & 0.0015681           & 0.910               \\
		Treatment 3         & 0.0009271           & 0.296               \\
		\bottomrule
	\end{tabular}
	\caption{Table caption.}
	\label{tab:example} % Unique label used for referencing the table in-text
\end{table}

Referencing \autoref{tab:example} in-text using its label.

\begin{table}[t] % Floating table, [t] tells LaTeX to place it at the top of the next available page
	\centering % Horizontally center the table on the page
	\begin{tabular}{L{0.15\textwidth} R{0.15\textwidth} R{0.15\textwidth}} % Specify column alignment with L{width}, C{width} and R{width} for fixed-width columns, or the default latex l, c and r for flexible-width columns
		\toprule
		\textbf{Treatments} & \textbf{Response 1} & \textbf{Response 2} \\
		\midrule
		Treatment 1         & 0.0003262           & 0.562               \\
		Treatment 2         & 0.0015681           & 0.910               \\
		Treatment 3         & 0.0009271           & 0.296               \\
		\bottomrule
	\end{tabular}
	\caption{Floating table.}
	\label{tab:floating} % Unique label used for referencing the table in-text
\end{table}

%------------------------------------------------

\section{Figure}\index{Figure}

Lorem ipsum dolor sit amet, consectetur adipiscing elit. Praesent porttitor arcu luctus, imperdiet urna iaculis, mattis eros. Pellentesque iaculis odio vel nisl ullamcorper, nec faucibus ipsum molestie. Sed dictum nisl non aliquet porttitor. Etiam vulputate arcu dignissim, finibus sem et, viverra nisl. Aenean luctus congue massa, ut laoreet metus ornare in. Nunc fermentum nisi imperdiet lectus tincidunt vestibulum at ac elit. Nulla mattis nisl eu malesuada suscipit.

\begin{figure}[H] % Use [H] to suppress floating and place the figure/table exactly where it is specified in the text
	\centering % Horizontally center the figure on the page
	\includegraphics[width=0.5\textwidth]{creodocs_logo.pdf} % Include the figure image
	\caption{Figure caption.}
	\label{fig:placeholder} % Unique label used for referencing the figure in-text
\end{figure}

Referencing \autoref{fig:placeholder} in-text using its label.

\begin{figure}[b] % Floating figure, [b] tells LaTeX to place it at the bottom of the next available page
	\centering % Horizontally center the figure on the page
	\includegraphics[width=\textwidth]{creodocs_logo.pdf} % Include the figure image
	\caption{Floating figure.}
	\label{fig:floating} % Unique label used for referencing the figure in-text
\end{figure}

%----------------------------------------------------------------------------------------

\stopcontents[part] % Manually stop the 'part' table of contents here so the previous Part page table of contents doesn't list the following chapters

%----------------------------------------------------------------------------------------
%	BIBLIOGRAPHY
%----------------------------------------------------------------------------------------

\chapterimage{} % Chapter heading image
\chapterspaceabove{2.5cm} % Whitespace from the top of the page to the chapter title on chapter pages
\chapterspacebelow{2cm} % Amount of vertical whitespace from the top margin to the start of the text on chapter pages

%------------------------------------------------

\chapter*{Bibliography}
\markboth{\sffamily\normalsize\bfseries Bibliography}{\sffamily\normalsize\bfseries Bibliography} % Set the page headers to display a Bibliography chapter name
\addcontentsline{toc}{chapter}{\textcolor{ocre}{Bibliography}} % Add a Bibliography heading to the table of contents

\section*{Articles}
\addcontentsline{toc}{section}{Articles} % Add the Articles subheading to the table of contents

\printbibliography[heading=bibempty, type=article] % Output article bibliography entries

\section*{Books}
\addcontentsline{toc}{section}{Books} % Add the Books subheading to the table of contents

\printbibliography[heading=bibempty, type=book] % Output book bibliography entries

%----------------------------------------------------------------------------------------
%	INDEX
%----------------------------------------------------------------------------------------

\cleardoublepage % Make sure the index starts on an odd (right side) page
\phantomsection
\addcontentsline{toc}{chapter}{\textcolor{ocre}{Index}} % Add an Index heading to the table of contents
\printindex % Output the index

%----------------------------------------------------------------------------------------
%	APPENDICES
%----------------------------------------------------------------------------------------

\chapterimage{orange2.jpg} % Chapter heading image
\chapterspaceabove{6.75cm} % Whitespace from the top of the page to the chapter title on chapter pages
\chapterspacebelow{7.25cm} % Amount of vertical whitespace from the top margin to the start of the text on chapter pages

\begin{appendices}

	\renewcommand{\chaptername}{Appendix} % Change the chapter name to Appendix, i.e. "Appendix A: Title", instead of "Chapter A: Title" in the headers

	%------------------------------------------------

	\chapter{Appendix Chapter Title}

	\section{Appendix Section Title}

	Lorem ipsum dolor sit amet, consectetur adipiscing elit. Aliquam auctor mi risus, quis tempor libero hendrerit at. Duis hendrerit placerat quam et semper. Nam ultricies metus vehicula arcu viverra, vel ullamcorper justo elementum. Pellentesque vel mi ac lectus cursus posuere et nec ex. Fusce quis mauris egestas lacus commodo venenatis. Ut at arcu lectus. Donec et urna nunc. Morbi eu nisl cursus sapien eleifend tincidunt quis quis est. Donec ut orci ex. Praesent ligula enim, ullamcorper non lorem a, ultrices volutpat dolor. Nullam at imperdiet urna. Pellentesque nec velit eget est euismod pretium.

	%------------------------------------------------

	\chapter{Appendix Chapter Title}

	\section{Appendix Section Title}

	Lorem ipsum dolor sit amet, consectetur adipiscing elit. Aliquam auctor mi risus, quis tempor libero hendrerit at. Duis hendrerit placerat quam et semper. Nam ultricies metus vehicula arcu viverra, vel ullamcorper justo elementum. Pellentesque vel mi ac lectus cursus posuere et nec ex. Fusce quis mauris egestas lacus commodo venenatis. Ut at arcu lectus. Donec et urna nunc. Morbi eu nisl cursus sapien eleifend tincidunt quis quis est. Donec ut orci ex. Praesent ligula enim, ullamcorper non lorem a, ultrices volutpat dolor. Nullam at imperdiet urna. Pellentesque nec velit eget est euismod pretium.

	%------------------------------------------------

\end{appendices}

%----------------------------------------------------------------------------------------

\end{document}
